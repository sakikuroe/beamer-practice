\documentclass[aspectratio=169, dvipdfmx, 11pt]{beamer}
\usepackage{tcolorbox}
\everymath{\displaystyle}

% Specify a design
\usetheme{Luebeck}
% Specify a color
\setbeamercolor{normal text}{bg=yellow!10!white}
\usecolortheme[RGB={100, 60, 100}]{structure}
% Specify a fontstyle
\usefonttheme{professionalfonts}
% Hide the navigation bar
\setbeamertemplate{navigation symbols}{}
% Display slide number
\setbeamertemplate{footline}[frame number]
% Specify a framestyle
\useoutertheme[subsection = false]{miniframes}
% Specify a fontstyle
\renewcommand{\kanjifamilydefault}{\gtdefault}

\definecolor{red}{rgb}{0.5, 0.0, 0.0}
\definecolor{green}{rgb}{0.0, 0.5, 0.0}
\definecolor{blue}{rgb}{0.0, 0.0, 0.5}

\title[]{TITLE}
\subtitle[]{sub title}
\author[]{KUROE, Saki}
\institute[]{Hogehoge University}
\date{\today}

\begin{document}
\begin{frame}{}
    \titlepage
\end{frame}

\begin{frame}{Table of Contents}
    \tableofcontents
\end{frame}

\section{完全加法族}
\begin{frame}{}
    \begin{tcolorbox}[
            colframe = red,
            colback = red!5!white,
            colbacktitle = red,
            coltitle = white,
            fonttitle = \bfseries,
            rightrule = 0pt,
            leftrule = 0pt,
            bottomrule = 0pt,
            arc = 0pt,
            title = Def 1.1. 完全加法族
        ]
        {
            \begin{tcolorbox}
                \begin{itemize}
                    \item $S$: 集合
                    \item $\mathcal{F} \in 2^{S}$
                \end{itemize}
            \end{tcolorbox}
            $\mathcal{F}$が完全加法族 \\
            $\overset{\text{def}}{\Longleftrightarrow}$
            \begin{enumerate}
                \item $\varnothing \in \mathcal{F}$
                \item $A \in \mathcal{F} \Rightarrow A^{c} \in \mathcal{F}$
                \item $\{A_{i}\}_{i = 1}^{\infty} \subset \mathcal{F} \Rightarrow \bigcup_{i = 1}^{\infty} A_{i} \in \mathcal{F}$
            \end{enumerate}
        }
    \end{tcolorbox}
\end{frame}

\begin{frame}{}
    \begin{tcolorbox}[
            colframe = red,
            colback = red!5!white,
            colbacktitle = red,
            coltitle = white,
            fonttitle = \bfseries,
            rightrule = 0pt,
            leftrule = 0pt,
            bottomrule = 0pt,
            arc = 0pt,
            title = Def 1.2. 可測空間
        ]
        {
            $\mathcal{(S, F)}$が可測空間\\
            $\overset{\text{def}}{\Longleftrightarrow}$
            \begin{itemize}
                \item $S$: 集合
                \item $\mathcal{F} \in 2^{S}$: 完全加法族
            \end{itemize}
        }
    \end{tcolorbox}
    \begin{tcolorbox}[
            colframe = red,
            colback = red!5!white,
            colbacktitle = red,
            coltitle = white,
            fonttitle = \bfseries,
            rightrule = 0pt,
            leftrule = 0pt,
            bottomrule = 0pt,
            arc = 0pt,
            title = Def 1.3. 可測
        ]
        {
            \begin{tcolorbox}
                $\mathcal{(S, F)}$: 可測空間
            \end{tcolorbox}
            $A$が可測
            $\overset{\text{def}}{\Longleftrightarrow}$
            $A \in \mathcal{F}$
        }
    \end{tcolorbox}
\end{frame}

\begin{frame}{}
    \begin{tcolorbox}[
            colframe = blue,
            colback = blue!5!white,
            colbacktitle = blue,
            coltitle = white,
            fonttitle = \bfseries,
            rightrule = 0pt,
            leftrule = 0pt,
            bottomrule = 0pt,
            arc = 0pt,
            title = Prop 1.4.
        ]
        {
            \begin{tcolorbox}
                $\mathcal{(S, F)}$: 可測空間
            \end{tcolorbox}
            $\{A_{i}\}_{i = 1}^{\infty} \subset \mathcal{F}
                \Rightarrow
                \bigcap_{i = 1}^{\infty} A_{i} \in \mathcal{F}$
        }
    \end{tcolorbox}
\end{frame}

\section{測度}
\begin{frame}
    \begin{tcolorbox}[
            colframe = red,
            colback = red!5!white,
            colbacktitle = red,
            coltitle = white,
            fonttitle = \bfseries,
            rightrule = 0pt,
            leftrule = 0pt,
            bottomrule = 0pt,
            arc = 0pt,
            title = Def 1.5. 測度
        ]
        {
            \begin{tcolorbox}
                \begin{itemize}
                    \item $\mathcal{(S, F)}$: 可測空間\\
                    \item $\mu: \mathcal{F} \rightarrow [0, \infty]$
                \end{itemize}
            \end{tcolorbox}
            $\mu$が測度\\
            $\overset{\text{def}}{\Longleftrightarrow}$
            \begin{itemize}
                \item $\mu(\varnothing) = 0 \,\,\,(\varnothing \in \mathcal{F})$
                \item 可算加法性:\\
                      $\{A_{i}\}_{i = 1}^{\infty}$ (disjoint) $\subset \mathcal{F},
                          \mu\left(\bigcup_{i = 1}^{\infty} A_{i}\right) = \sum_{i = 1}^{\infty}\mu(A_{i})$
            \end{itemize}
        }
    \end{tcolorbox}
\end{frame}
\end{document}